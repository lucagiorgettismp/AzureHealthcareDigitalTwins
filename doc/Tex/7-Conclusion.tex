\chapter{Conclusione}

Al termine dell'elaborato, possiamo ritenerci più che soddisfatti del risultato ottenuto, essendo riusciti a raggiungere gli obiettivi prefigurati.\newline

L'utilizzo delle tante tecnologie, non conosciute o viste in maniera superficiale all'inizio del lavoro (\textit{Unity}, \textit{Hololens} e \textit{Azure}), è stato sicuramente il primo scoglio da superare, ed effettivamente, da una parte ha rallentato la tabella di marcia ma allo stesso tempo ci ha permesso di acquisire nuove conoscenze. Queste potranno rivelarsi utili in futuro, vista che si tratta di tecnologie e piattaforme sempre più studiate ed utilizzate.\newline

Inizialmente avevamo fatto un'indagine per lo sviluppo dei digital twins su \textit{Eclipse Ditto} e \textit{Hono}, tentativo che non ha portato a nulla di concreto per via del malfunzionamento del sevizio di test online e dell'elevato utilizzo delle risorse per farlo eseguire sulle nostre macchine. Il problema ci ha fatto apprezzare quanto il migrare ad una soluzione come \textit{Azure}, di tipo PaaS (Platform as a Service) abbia semplificato la configurazione dei servizi e lo sviluppo su di essi.
\newline

A causa dell'inesperienza con le tecnologie utilizzate, con il senno di poi avremmo considerato con più attenzione aspetti che avevamo trascurato inizialmente. Ad esempio attribuire sin da subito agli oggetti di \textit{Unity} la giusta dimensione per essere visualizzati correttamente tramite \textit{Hololens}, piuttosto che l'attenzione alle performance della nostra applicazione.\newline

Nel complesso, nonostante le difficoltà, siamo riusciti ad implementare ed integrare tutto quello che avevamo previsto: il simulatore del monitor a parametri vitali, il client per la creazione del paziente, l'infrastruttura \textit{Azure Digital Twins} e tutti gli ologrammi appartenenti alla realtà aumentata. \newline

Uno dei problemi più spinosi è stato il \textit{porting} su piattaforma \textit{Hololens} delle librerie testate ed utilizzate su Windows, cosa che ritenevamo più fluida. Alcune di queste librerie hanno invece dato problemi solamente a \textit{runtime}, rallentando lo sviluppo. 
Inoltre la piattaforma \textit{Unity} si è dimostrata ostica nel riuscire a lavorare con più macchine sulla stessa scena, per via del modo in cui vengono assegnati gli id degli oggetti, costringendoci a ripetere il binding di questi ad ogni modifica. \newline

In conclusione, siamo certi che la tecnologia dei \textit{Digital Twins} sarà sempre più pervasiva visto il grande potenziale che può offrire: sia in termini di contributo sia sull'ampia scelta di settori in cui può essere utilizzata: dal settore ospedaliero a quello industriale passando per qualsiasi settore ingegneristico, per via degli innumerevoli vantaggi che può portare al netto dei costi di implementazione.