\chapter{Design di Dettaglio}
 \label{chap:detailed-design}
In questo capitolo verrà illustrata l'organizzazione del codice per i componenti dell'architettura, i pattern utilizzati e le principali scelte implementative.

\section{Organizzazione del Codice}

Il progetto è consultabile al seguente \href{https://github.com/lucagiorgettismp/AzureHealthcareDigitalTwins}{\textit{repository}} di GitHub. Il repository è organizzato nelle seguenti directory:

\begin{itemize}

    \item \textbf{VitalSignsMonitorSimulator}: contiene la soluzione C\# del simulatore;
    
    \item \textbf{DTDLModels}: contiene i modelli \texttt{.json} per la definizione dei digital twins;
    
    \item \textbf{AzureTools}: contiene la soluzione C\# di tutte le azure function implementate;

    \item \textbf{HololensClient}: contiene il progetto Unity per la parte di mixed reality;
        
    \item \textbf{doc}: contiene la documentazione in formato \texttt{.latex} e \texttt{.pdf}.
\end{itemize}

\subsection{Simulatore}
Il simulatore è un progetto .NET scritto in C\# che contiene tre moduli:
\begin{itemize}
    \item \textbf{Client}:
    \item \textbf{Simulator}:
    \item \textbf{Common}: 
\end{itemize}

\subsection{Azure Digital Twins}

\subsubsection{Modelli}

\subsubsection{Azure function}

\subsection{Mixed Reality - Hololens}

\section{Pattern utilizzati}