\chapter{Introduzione}
Il progetto si pone come obbiettivo la realizzazione di un applicativo basato su digital twins contestualizzato in ambito sanitario. L'idea è quella di poter creare un'\textit{anima digitale} (o \textit{digital twin}) di dispositivi fisici (detti anche \textit{phisical assets}), utilizzati da medici e infermieri durante gli interventi in sala operatoria. Il digital twin quindi, permetterà di realizzare una sua rappresentazione digitale grazie all'utilizzo della realtà aumentata. \newline \newline Uno scenario tipico è quello di poter creare un ologramma di un monitor a parametri vitali che ogni membro del team dotato di un visore, può visualizzare nello spazio. Grazie all'utilizzo di appositi occhiali (\href{https://www.microsoft.com/it-it/hololens}{\textit{Hololens}}), il personale di sala operatoria potrà vedere ed interagire con l'ologramma associato ad uno specifico \textit{asset}, in maniera indipendente dal posizionamento dell'\textit{asset} fisico e dei i suoi colleghi. Ovviamente i dati visualizzati nei dispositivi fisici saranno aggiornati in \textit{real time} anche nell'ologramma
\newline \newline Le tecnologie utilizzate saranno quindi legate all'implementazione dei digital twins, dei rispettivi \textit{assets} fisici (che in questo caso verranno simulati) e alla sua rappresentazione digitale nella spazio (\textit{mixed reality}) che ne permette appunto l'interazione con l'uomo.