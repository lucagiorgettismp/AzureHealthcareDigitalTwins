\chapter{Introduzione}
Il progetto si pone come obbiettivo la realizzazione di un applicativo basato su digital twins contestualizzato in ambito sanitario. L'idea è quella di poter creare un'\textit{anima digitale} (o \textit{digital twin}) di dispositivi fisici (detti anche \textit{phisical asset}), utilizzati da medici e infermieri durante gli interventi in sala operatoria. Successivamente, il digital twin permette di realizzare una sua rappresentazione digitale grazie all'utilizzo della realtà aumentata. Uno scenario tipico è quello di poter creare un ologramma di un monitor a parametri vitali che ogni membro del team può visualizzare nella spazio. Quello che si viene a creare è una connessione dell'\textit{asset} con il suo oggetto 3D così che ogni modifica, da parte dell'uomo, porti ad un aggiornamento delle informazioni in entrambi le direzioni. Grazie all'utilizzo di appositi occhiali (\textit{Hololens}, \textit{Magic leap}), ogni infermiere potrà vedere l'ologramma associato ad uno specifico \textit{asset}, in maniera indipendente dal posizionamento di quello fisico e da tutti i suoi colleghi. Ovviamente i dati visualizzati nei dispositivi fisici saranno aggiornati in \textit{real time} anche nell'ologramma.\newline \newline Le tecnologie utilizzate saranno quindi legate all'implementazione dei digital twins, dei rispettivi assets fisici (che in questo caso verranno simulati) e alla sua rappresentazione digitale nella spazio (mixed reality) che permette appunto l'interazione con l'uomo.